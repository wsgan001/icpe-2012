
\section{Context/Problem Statement} \label{sec:context}
This section further details the centralized access control architecture as well as its desirable feature (synergy, reconfigurability) and its weakest ones (performance bottlenecks). 
Managing access control policies is one of the most challenging issues faced by organizations. Frequent changes in policy-based systems may be required to meet business needs. 
A policy-based system has to handle some specific requirements like role swapping when employees are given temporary assignments, changes in the policies and procedures, 
new assets, users and job positions in the organization.

\subsection{Centralization of Access Control Architectures}
To enable high reconfigurability, the access control policy is traditionally modeled, analyzed and implemented as a separate component 
encapsulated in a PDP. This leads to the centralized architecture presented in Figure \ref{pep-pdp}, in which one PDP is responsible for granting/denying the accesses that are requested. 
This centralized architecture is a simple solution to easily handle changes in policy-based systems by having an immediate translation from the policy author to the PDP. Reconfiguration consists of modifying the PDP accordingly to 
the changes in the access control policy. The separation between the PEP and the PDP simplifies policy management across many heterogeneous systems and limits
potential risks arising from incorrect policy implementation, when the policy is hardcoded inside the business logic.

\begin{figure}[!h]
\begin{center}
\includegraphics[width=9cm, height=8cm]{business-logic}
\caption{Access Control Request Processing}
\label{pep-pdp}
\end{center}
\end{figure}

\subsection{Centralization: A Threat for performances}
In such a system (Figure \ref{pep-pdp}), when the service execution requires an access control, the PEP calls the PDP to 
retrieve an authorization decision based on the PDP encapsulated policy. This authorization decision is made through the evaluation of rules in the policy. 
Subsequently, an authorization decision (permit/deny) is returned to the PEP. When a huge number of access requests are sent to the PDP, two bottlenecks cause a degradation of performances:
\begin{itemize}
 \item 1) all the access requests have to be managed through the same input channel of the PDP. 
 \item 2) the centralized PDP computes an access request by searching which rule is applicable among all the rules it contains.
\end{itemize}

The execution time for the treatment of a request is thus strongly related to:
\begin{itemize} 
\item the number of access rules the PDP contains as well as to the ordering of the rules into a PDP \cite{clustering}.
\item the workload (number of requests).
\end{itemize}
The execution time to treat one request depends on the size (number of rules) the PDP contains. For a given PDP size, the execution time  to treat requests increases linearly with the workload (number of requests). 
Our hypothesis is: the more rules a PDP contains, the higher the slope gradient of the execution time submitted to an increasing workload. 
 \textit{Hypothesis 1} validity is studied in section \ref{sec:experiment}.
As a consequence, one possibility to increase performances consists in splitting the centralized PDP into PDPs of smaller sizes. 


\subsection{Centralization allows a good synergy between PEPs and PDP}
Centralization offers a desirable feature by simplifying the routing of requests to the right PDP. 
Figure \ref{model} illustrates the model of the access control architecture. In this model, a set of business processes, which comply to users' needs, are encapsulated in a given business logic 
which is enforced by multiple PEPs. Conceptually, the decision is decoupled from the enforcement and involves a decision making process in which each PEP 
interacts with one single PDP. The keypoint concerns the cardinality linking PEPs to the PDP. While a PDP is potentially linked to many PEPs, any PEP is strictly linked to exactly one 
PDP (which is unique in the centralized model). 
Since there is only one PDP, the requests are all routed to this unique PDP. This means that no particular treatment is required to map a given PEP in the business logic to 
the corresponding PDP, embedding the requested access rule. Another advantage of this many-to-one association is the clear traceability between what has been specified by the 
policy at the decision level and the internal security mechanisms enforcing this policy at the business logic level. In such setting, 
when access control policies are updated or removed, the related PEPs can be easily located and removed. Thus the application is updated synchronously 
with the policy changes. We call this desirable property \textit{synergy} of the access control architecture: an access control architecture is said to be \textit{synergic} if any PEP always sends its requests  to the same PDP. 
As a consequence, splitting the centralized PDP into PDPs of smaller sizes may break this "synergy". In the following, we will consider various ways to split a centralized PDP into smaller PDPs. 
The related \textit{hypothesis 2} is: with comparable PDP sizes, the overall system will be more performant when the architecture is synergic. This hypothesis is questioned in 
section \ref{sec:experiment}.

\subsection{Tradeoff for refactoring a centralized architecture into a decentralized one}

As a synthesis for this section, the following facts are taken into account:
\begin{itemize}
 \item Access control architectures are centralized with a unique PDP
\item Centralization eases reconfiguration of access control policy
\item Centralization threatens performances
\item Direct mapping from any PEP to only one PDP makes the access control architecture "synergic"
\item A synergic system facilitates PEPs request routing and eases access control policy updates

\end{itemize}

The goal of this paper is to propose systematic ways to improve performance by refactoring the centralized model into a decentralized version, with multiple PDPs. The resulting
 architecture must be functionally equivalent and should not impact the desirable properties of the centralized model, namely reconfigurability and synergy.
Automating the transformation from a centralized to a decentralized architecture is required to preserve reconfigurability. With automation, we can still reconfigure the centralized policy, 
and then automatically refactor the architecture. Automated refactoring is thus a viable solution for having high reconfigurability.  
However, refactoring the architecture by splitting the centralized PDP into smaller one may break the initial synergy. This phenomenon is studied in the empirical study of section 
\ref{sec:experiment} together with \textit{hypothesis 2}. 
We describe the XACML Language since it is the standard language used to implement a PDP.


\begin{figure}[!h]
\begin{center}
\includegraphics[height=5.5cm,width=8.5cm]{model}
\caption{The Access Control Model}
\label{model}
\end{center}
\end{figure}

\subsection{XACML Access Control Policies and Performance Issues}

In this paper, we focus on access control policies specified in the eXtensible Access Control Modeling Language (XACML) \cite{sunxacml}.
XACML is an XML-based standard policy specification language that defines a syntax of access control policies and
request/response. \\XACML enables policy authors to externalize access control policies for the sake of interoperability since access control policies can be designed 
independently from the underlying programming language or platform. Such flexibility enables to easily update access control policies to comply with new requirements.

An XACML is constructed as follows.
A \CodeIn{policy set} element consists of a sequence of \CodeIn{policy elements}, a combining algorithm, and
a \CodeIn{policy target} element. A \CodeIn{policy element} is expressed through a target, a set of \CodeIn{rules}, and a rule combining algorithm. 
A \CodeIn{target} element consists of the set of resources, subjects, and actions to which a rule is applicable. A \CodeIn{rule} consists of a 
\CodeIn{target} element, a \CodeIn{condition} element, and an \CodeIn{effect}. A \CodeIn{condition} element is a boolean expression that specifies the
environmental context (e.g., time and location restrictions) in which the rule applies.
Finally, an \CodeIn{effect} is the rule's authorization decision, which is either permit or deny.

Given a request, a PDP evaluates the request against the \CodeIn{rules} in the policy by matching resources, subjects, actions, and condition in the request.
More specifically, an XACML request encapsulates attributes, which define which subject requests to take action on which resource in which
condition (e.g., subject Bob requests to borrow a book).
%This can be under/not a condition.
Given a request, that satisfies \CodeIn{target} and \CodeIn{condition} elements in a rule, the rule's effect
is taken as the decision.
If the request does not satisfy \CodeIn{target} and \CodeIn{condition} elements in any rule, its response yields the ``NotApplicable'' decision.

When more than one rule is applicable to a request, the combining algorithm helps determine which rule's effect can be finally given as the decision for the request.
For example, given two rules, that are applicable to the same request and provide different decisions,
the permit-overrides algorithm prioritizes a permit decision over the other decisions.
More precisely, when using the permit-overrides algorithm, the policy evaluation produces one of the following three decisions: 

\begin{itemize}
\item Permit if at least one permit rule is applicable for a request.
\item Deny if no permit rule is applicable and at least one deny rule is applicable for a request.
\item NotApplicable if no rule is applicable for a request.
\end{itemize}

Figure \ref{figur1} shows a simplified XACML policy that denies subject Bob to borrow a book.

%\fontsize{5}{5}
\begin{figure}[!h]
\begin{center}
\includegraphics[width=8.6cm]{xacml}
\caption{XACML Policy Example}
\label{figur1}
\end{center}
\end{figure}

 
\Comment{ 
\begin{figure}[!h]%{t}
%\begin{figure}[firstnumber=100]
\begin{CodeOut}
\begin{alltt}
\small
 1 <PolicySet PolicyId="\textbf{An Example Policy Set}" PolicyCombAlgId="\textbf{Permit-overrides}">
 2 <Target/> 
 3  <Policy PolicyId="\textbf{An Example Policy}" RuleCombAlgId="\textbf{Permit-overrides}">
 4   <Target/>
 5    <Rule RuleId="\textbf{1}" Effect="\textbf{Deny}">
 6      <Target>
 7        <Subjects><Subject> \textbf{Bob} </Subject></Subjects>
 8        <Resources><Resource> \textbf{BOOK} </Resource></Resources>
 9        <Actions><Action> \textbf{BORROW} </Action></Actions>
10      </Target>
11	    <Condition>
12        <AttributeValue> \textbf{DEFAULT} </AttributeValue>
13      </Condition>
14    </Rule>
15      <!-- A final, "fall-through" rule that always Denies -->
16    <Rule RuleId="\textbf{FinalRule}" Effect="\textbf{Deny}"/>
17  </policy>
18 </policySet>
\end{alltt}
\end{CodeOut}
\vspace*{-3.0ex} \caption{An XACML Policy Example}
 \label{figur1}
\end{figure}
}
 

%\centering
%\figure[DREF metamodel\label{fig:drefMM}]
%        {\includegraphics[width=0.5\textwidth]{figure/drefMM}}
%\figure[SETER Process: Security Testing for Resilient Systems 
%\label{fig:seter}] {\includegraphics[width=0.49\textwidth]{figure/seter}}
%\caption{DREF and SETER: Conceptual and Operational Frameworks for evaluating resilient systems}
%\end{figure*}


Recently, an XACML policy becomes more complex to handle increasing complexity of organizations in terms of structure, relationships, activities, and access control requirements. In such a situation, the policy often consists of a large number of rules to specify policy behaviors for various resources, users and actions in the organizations.
In policy-based systems, policy authors manage centralized a single PDP loaded with a single policy to govern all of system resources. 
However, due to a large number of rules for evaluation, this centralization raises performance concerns related to request evaluation time for XACML access control policies and may 
degrade the system efficiency and slow down the overall business processes. 

We present following three main factors, that may cause to degrade XACML request evaluation performance: 

\begin{itemize}
\item An XACML policy may contain various attribute elements including \CodeIn{target} elements. Retrieval of
attribute values in the \CodeIn{target} elements for request evaluation may increase the evaluation time.
\item A \CodeIn{policy set} consists of a set of policies. Given a request, a PDP determines the final authorization decision (i.e., effect) of the whole \CodeIn{policy set} after combining all the applicable rules' decisions according to the request.
Computing and combining applicable rules' decisions contributes to increase the evaluation time.
\item \CodeIn{Condition} elements in rules can be complex because these elements are built from an arbitrary nesting of non-boolean functions and attributes. 
In such a situation, evaluating \CodeIn{condi-\\tion} elements may slow down request evaluation time.
\end{itemize}

