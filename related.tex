\section{Related Work} \label{sec:related}
Some previous contributions have addressed access control policies performance issues, In \cite{clustering}, the authors have proposed an approach for policy evaluation based on a 
clustering algorithm that reorders rules and policies within the policy set so that the access to applicable policies is faster, their categorization is based on
 the subject target element. Their technique requires identifying the rules that are frequently used. Our approach follows a different strategy and does not require knowing which 
rules are used the most. In addition, the rule reordering is tightly related to specific systems. If the PDP is shared between several
 systems, their approach could not be applicable since the most ``used'' rules may vary between systems. \\
In \cite{decomposition}, 
the authors decomposed the global XACML policy into local policies related to collaborating parties, the local policies 
are sent to corresponding PDPs. The request evaluation is based on local policies by considering the relationships among local
 policies. In their approach, the optimization is based on storing the effect of each rule and each local policy for 
a given request. Caching decisions results is then used to optimize evaluation time for an incoming request. However the authors have 
not provided experimental results to measure the efficiency of their approach when compared to the traditional architecture.  

While the previous approaches have focused on the PDP component to optimize the request evaluation, The authors in 
\cite{XACMLstructure}, addressed this problem by analyzing rule location on XACML policy and requests at design level so that the relevant rules for the request are 
accessed faster on evaluation time. \\
The current contribution brings new dimensions to our previous work on access control \cite{Xengine} \cite{testcase} \cite{models}.
We have focused particularly in \cite{Xengine} on performance issues addressed with XACML policies evaluation and we have proposed an 
alternative solution to brute force searching based on an XACML policy conversion to a tree structure to minimize the request evaluation time. 
Our previous approach involves a refactoring process that transforms the global policy into a decision diagram converted into 
forwarding tables. In the current contribution, we introduce a new refactoring process that involves splitting the policy into smaller sub-policies. Our 
two refactoring processes can be combined to dramatically decrease the evaluation time. 
