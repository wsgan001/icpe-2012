\section{Introduction} \label{sec:introduction}

Access control mechanisms regulate which users
could perform which actions on what system resources based on access control policies.
Access control policies (i.e., policies in short) are based on various access control models such as Role-Based Access Control (RBAC)~\cite{ferraiolo:rbac}, Mandatory Access Control (MAC)~\cite{mac}, Discretionary Access Control (DAC)~\cite{dac}, and Organization-Based Access Control (OrBAC)~\cite{orbac}.
Access control policies are specified in various policy specification languages such
as the eXtensible Access Control Markup Language (XACML) \cite{sunxacml}
and Enterprise Privacy Authorization Language (EPAL) \cite{epal}.
%In this paper, we consider access control policies specified in
%the eXtensible Access Control Markup Language (XACML) \cite{sunxacml}.
In a policy, a rule specifies actions (e.g., read) that subjects (e.g., students) can take on resources (e.g., grades) if required conditions are met.



A policy-based system is a system that interacts with user-specified policies
to control users' access on resources. A policy-based system allows policy authors to define rules in a policy.
In the context of policy-based systems,
access control architectures are often designed with respect to a popular
architectural concept that separates Policy Enforcement Points (PEPs) from a Policy Decision Point (PDP) \cite{separation}. More specifically, a PEP is located inside an 
application's code (i.e., business logic of the system).
Business logic describes functional algorithms to govern information exchange between access control decision logic and a user interface (i.e., presentation).

Given requests (e.g., student $A$ requests to read her grade resource $B$) formulated by the PEP, the PDP evaluates the requests and returns their responses (e.g., permit or deny) by evaluating these requests
against rules in a policy. An important benefit of the architecture is to facilitate managing access rights in a fine-grained way by
decoupling the business logic from the access control decision logic, which can be standardized.

However, this architecture may cause performance degrade
especially when policy authors maintain a single policy with a large number of rules to regulate a whole system's resources.
Various factors such as complex and dynamic behaviors of organizations and the growth of organizations's assets may increase the
number of rules in the policy \cite{policymanagement}.
Consider that the policy is centralized into \emph{only} one single PDP.
%With centralization of a single PDP, policy authors
%Centralization facilitates managing and maintaining the policy,
%which is encapsulated into a single PDP.
The PDP evaluates requests (issued by PEPs) against
the large number of rules in the policy in real-time.
Such centralization can be a major factor for degrading performance as our previous work showed that the large number of rules is a challenge for efficient request evaluation~\cite{Xengine}.
This performance bottleneck issue may impact service
availability as well, especially when dealing with a huge number of requests within a short time.


In order to address this performance bottleneck issue,
we propose an approach to refactoring policies automatically to significantly reduce
request evaluation time.
As manual refactoring is tedious and
error-prone, an important benefit of our automated approach is to reduce significant human efforts as well as
improving performance.
Our approach includes two techniques: (1) refactoring an access control policy (handled by single PDP) into its corresponding multiple access
control policies each with a smaller number of rules (handled by multiple PDPs),
and (2) refactoring PEPs with regards to the refactored PDPs while preserving architectural property that a single PDP is triggered by a given PEP at a time.\\

In the first technique, our approach takes a splitting criterion and an original global policy (i.e., a policy governing all of access rights in the system) as an input, and returns a set of
corresponding sub-policies, each of which consists of a smaller number of rules.
This refactoring involves grouping rules in the global policy into several subsets based on the splitting criterion.
More specifically, we propose a set of splitting criteria to
refactor the global policy into smaller policies.
A splitting criterion selects and groups the rules of the overall PDP into specific PDPs.
Each criterion-specific PDP encapsulates a sub-policy that represents a set of rules that share a combination
of attribute elements (Subject, Action, and/or Resource).

In the second technique, our approach aims at preserving the architectural property, which represents that only single PDP is triggered by a given PEP at a time.
Our approach refacors PEPs according to multiple PDPs loaded with sub-policies while complying policy behaviors of the centralized architectures
and preserving the architectural property. More specifically, given a request, each PEP should be mapped to a PDP loaded with a policy, 
which includes a set of rules to be applicable for the request.
Therefore, our refactoring maintains architectural property of the centralized architectures in policy-based systems.



%We conducted an evaluation on three subjects of real-life JAVA program, each of which interact
%with access control policies.
%Our evaluation results show that our proposed approach
%preserves the initial architectural model of the subjects in terms of interaction between the business logic and its corresponding
%rules in the policy. Our evaluation results show that our approach
%is efficient in terms of reducing request evaluation time by up to nine times.



%Given a set of requests, we then compare evaluation time of the requests against the original policy and a set of sub-polices
%based on the proposed splitting criteria.

We collect three subjects of real-life Java programs, each of which interacts
with access control policies.

The policies consist of a large number of rules. The largest one has 1760 rules, whose corresponding request evaluation faces performance degrade.
The policies are specified in XACML \cite{xacml}.
%Theses three polices consist of comparatively large number of rules 720, 945, 1760 rules, respectively, which
%is comparatively large number of rules.
XACML is an XML-based policy specification language popularly used for web-based applications and services.
While our subjects are based on XACML policies,
our approach could be
applicable to any software system that interacts with policies specified in other
policy specification languages. As a large number of rules are performance bottlenecks for request evaluation
against given policies, our approach could reduce the number of rules in a systemic way for request evaluation against the policies to improve performance.



We conduct an evaluation to show performance of our approach in terms of request evaluation time.
We leverage two PDPs to measure request evaluation time. The first one is the
Sun PDP implementation~\cite{oasis}, which is a popular open source PDP and the second one is
XEngine \cite{Xengine}, which transforms an original policy
into its corresponding policy in a tree format by mapping attribute values with numerical values.
Our evaluation results also show that our approach
preserves the policy behaviors of the centralized architecture and the architectural property. Our evaluation results show that our approach
is efficient in terms of reducing request evaluation time by up to nine times.


%Our contribution in this paper goes in the same research direction as it aims to reduce the request evaluation time by refactoring the policy.
%Our evaluation results show that decision making time is reduced up to nine times with split policies if compared to its original global policy
%and that we have a considerable performance improvement, if the policies resulting from our refactoring process are evaluated
%with XEngine rather than Sun PDP.

This paper makes the following three main contributions:
\begin{itemize}
\item We propose an automated approach that refactors a single global policy into policies each with a smaller number of rules. This
refactoring helps improve performance of request evaluation time.
\item We propose a set of splitting criteria to help refactor a policy in a systematic way. Our proposed splitting criteria do not alter policy behaviors of the centralized architectures.
\item We conduct an evaluation on three Java programs interacting with XACML policies. We measure performance in terms
of request evaluation time.
%We compare our approach with an approach based on the global policy in terms of efficiency.
Our evaluation results show that our approach achieves more than nine times faster than that of the centralized access control architectures in terms of request evaluation time.
\end{itemize}


The remainder of this paper is organized as follows. Section~\ref{sec:context} introduces concepts related to our research problem addressed in this paper.
Section~\ref{sec:approach} presents the overall approach.
Section~\ref{sec:experiment} presents evaluation results and discusses the effectiveness of our approach. Section~\ref{sec:related} discusses related work.
Section~\ref{sec:conclusion} concludes this paper and discuses future research directions.

