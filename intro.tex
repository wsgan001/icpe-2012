
\section{Introduction} \label{sec:introduction}

Access control mechanisms regulate which users
could perform which actions on system resources based on access control policies.
Access control policies (i.e., policies in this paper) are based on various access control models such as Role-Based Access Control (RBAC)~\cite{ferraiolo:rbac}, mandatory access control (MAC)~\cite{mac}, discretionary access control (DAC)~\cite{dac}, and Organisation-based access control (OrBAC)~\cite{orbac}.
In this paper, we consider access control policies specified in 
the eXtensible Access Control Markup Language (XACML) \cite{sunxacml}. XACML is a popularly used XML-based language to specify rules 
in a policy. A rule specifies actions (e.g., access) that subjects (e.g., student) can take on resources (e.g., grades) if required conditions are met.

In the context of policy-based systems, access control architectures are often built with respect to a popular
architectural concept, which separates Policy Enforcement Points (PEP) from Policy Decision Point (PDP) \cite{separation}. More specifically, the PEP is located inside an application code (i.e., business logic of the system).
Given requests (e.g., can student $A$ access her grade resource $B$) formulated by the PEP. the PDP evaluates the requests and returns their responses (e.g., permit or deny) by evaluating these requests 
against rules in a policy. An important benefit of this architecture is to facilitate managing access rights in a fine-grained way by 
decoupling the business logic from the access control decision logic, which can be standardized. 

However, this architecture may cause performance degrade
especially, when policy authors maintain a single policy with a large number of rules to regulate a whole system resources.
Various factors such as complex and dynamic behaviors of organizations and the growth of organizations's assets may increase the 
number of rules in the policy \cite{policymanagement}. 
Consider that the policy is centralized into \emph{only} one single PDP.
%With centralization of a single PDP, policy authors 
%Centralization facilitates managing and maintaining the policy, 
%which is encapsulated into a single PDP.
The PDP evaluates requests (issued by PEPs) against
the large number of rules in the policy in real-time.
Such centralization can be a pitfall for degrading performance as our previous work showed that the large number of rules is critical for efficient request evaluation~\cite{Xengine}.
This performance bottleneck issue may impact service 
availability as well, especially in face of explosive number of requests.


In order to address this performance bottleneck issue,
we propose an approach to refactor policies automatically to significantly reduce
request evaluation time.
As manual refactoring is tedious and
error-prone, an important benefit of our automated approach is to reduce significant human efforts as well as
improves performance.
Our approach includes two facets: (1) refactor an access control policy (handled by single PDP) into its corresponding multiple access
control policies with smaller number of rules (handled by multiple PDPs),
and (2) refactor PEPs with regards to refactored PDPs while preserving architectural property that a single PDP is triggered by a given PEP at a time.\\

In the first facet, our approach takes a splitting criterion and an original global policy (i.e., a policy governing all of access rights in the system) as an input, and returns a set of 
corresponding sub-policies, each of which consists of smaller number of rules.
This refactoring involves grouping rules in the global policy  into several subsets based on splitting criteria.
More specifically, we propose a set of splitting criteria to
refactor the global policy into smaller policies.
A splitting criterion selects and groups the access rules of the overall PDP into specific PDPs.
Each criterion-specific PDP encapsulates a sub-policy that represents a set of rules that share a combination
of attribute elements (Subject, Action, and/or Resource).

In the second facet, our approach aims at preserving architectural property where a single PDP is triggered by a given PEP at a time.
Our approach refacors PEPs according to multiple PDPs loaded with sub-policies while complying behaviors of initial architectures in policy-based systems. More specifically, each PEP should be mapped to a PDP with a set of relevant rules for given requests issued by the PEP.
Therefore, our refactoring maintains behaviors of initial architectures in policy-based systems.
 


%We conducted an evaluation on three subjects of real-life JAVA program, each of which interact
%with access control policies. 
%Our evaluation results show that our proposed approach
%preserves the initial architectural model of the subjects in terms of interaction between the business logic and its corresponding
%rules in the policy. Our evaluation results show that our approach
%is efficient in terms of reducing request evaluation time by up to nine times. 



%Given a set of requests, we then compare evaluation time of the requests against the original policy and a set of sub-polices 
%based on the proposed splitting criteria.

We collect three subjects of real-life JAVA program, each of which interact
with access control policies. 
We conduct an evaluation to show performance of our approach in terms of request evaluation time.
We leverage two PDPs to measure request evaluation time; First one is
Sun PDP implementation~\cite{oasis}, which is a popular open source PDP and Second one is 
XEngine \cite{Xengine}, which transforms an original XACML policy
into its corresponding policy in a tree format by mapping attribute values with numerical values.
Our evaluation results show that our approach
preserves the initial architectural model of the subjects in terms of interaction between the business logic and its corresponding
rules in the policy. Our evaluation results show that our approach
is efficient in terms of reducing request evaluation time by up to nine times. 


%Our contribution in this paper goes in the same research direction as it aims to reduce the request evaluation time by refactoring the policy.
%Our evaluation results show that decision making time is reduced up to nine times with split policies if compared to its original global policy 
%and that we have a considerable performance improvement, if the policies resulting from our refactoring process are evaluated
%with XEngine rather than Sun PDP.

This paper makes the following three main contributions:
\begin{itemize}
\item We propose an automated approach that refactors a single global policy into policies with smaller number of rules. This
refactoring helps improve performance of request evaluation time.
\item We propose a set of splitting criteria to help refactor a policy in a systematic way. Our proposed splitting criteria does not alter behaviors of initial access control architectures.
\item We conduct an evaluation on three Java programs interacting with XACML policies. We measure performance in terms
of request evaluation time. 
%We compare our approach with an approach based on the global policy in terms of efficiency. 
Our evaluation results show that our approach achieves more than nine times faster than that of initial access control architectures in terms of request evaluation time.
\end{itemize}


The remainder of this paper is organized as follows: Section~\ref{sec:context} introduces concepts related to our research problem addressed in this paper.
Section~\ref{sec:approach} presents the overall approach. 
Section~\ref{sec:experiment} presents evaluation results and discusses the effectiveness of our approach. Section~\ref{sec:related} discusses related work.
Section~\ref{sec:conclusion} concludes this paper and discuses future research directions.

