\section{Conclusion and Future Work} \label{sec:conclusion}

In this paper, we have tackled the performance issue in access control decision making mechanism 
and we have proposed an
 automated refactoring  process  
that enables to reduce access control policies evaluation time up to 9 times.
Our approach has been applied to XACML policies and it can be generalized to policies in other policy specification
languages (such as EPAL). 
To support and automate the refactoring process, we have designed and implemented the ``PolicySplitter'' tool,
which transforms a given policy into small ones,
according to a chosen splitting criterion.
The obtained results have shown a significant gain in evaluation time when using any of the splitting criteria.
 The best gain in performance is reached by
the criterion that respects the synergy property. This plead in favor of a refactoring process that takes into
 account, the way PEPs are scattered inside the system business logic. 
In this work, we have easily identified the different PEPs since we know exactly how our system functions are
 implemented and thus how PEPs are organized inside the system. In a future work, we propose to automatically identify the different PEPs of a given application. 
This technique is an important step that is complementary to this paper approach, since it enables
 knowing how PEPs are organized in the system and thus allows to select automatically the most 
suitable splitting criterion for a given application. 