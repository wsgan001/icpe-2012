\section{Conclusion and Future Work} \label{sec:conclusion}

In this paper, we have tackled the performance issue in access control decision making mechanism.
We have proposed an
 automated refactoring  process  
that enables to reduce request evaluation time.
Our approach has been applied on three JAVA projects interacting with XACML policies.
To support and automate the refactoring process, we have designed and implemented a tool, called PolicySplitter,
which transforms a given policy into multiple policies according to a chosen splitting criterion.

Our evaluation results have shown that our approach gain a significant performance improvement
in terms of request evaluation time.
Our evaluation results have shown that our proposed approach
preserves the initial architectural model of the subjects in terms of interaction between the business logic and its corresponding
rules in the policy. Our evaluation results have shown that our approach
is efficient in terms of reducing request evaluation time by up to nine times. 

% The best gain in performance is reached by
%the criterion that respects the synergy property. This plead in favor of a refactoring process that takes into
% account, the way PEPs are scattered inside the system business logic. 
%In this work, we have easily identified the different PEPs since we know exactly how our system functions are
% implemented and thus how PEPs are organized inside the system.

%In our future work, we plan to develop an approach to automatically identify PEPs. 
%This technique is an important step that is complementary to this paper approach, since it enables
% knowing how PEPs are organized in the system and thus allows to select the most 
%suitable splitting criterion for a given application. 
% and it can be generalized to policies in other policy specification
%languages (such as EPAL). 
