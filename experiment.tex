%\section{Experiment}\label{sec:experiment}

\section{Evaluation} \label{sec:experiment}
We carried out our evaluation on a desktop PC, running Ubuntu 10.04 with a Core i5, 2530 Mhz processor, and 4 GB of RAM. 
We have implemented a tool, called \CodeIn{PolicySplitter} to split the policies according to given splitting criteria automatically.
The tool is implemented in Java and is available for download from \cite{splitter}.

\subsection{Objectives and Measures}
What we want to achieve in the evaluation section is to answer the following research questions:
\begin{enumerate}
\item \textbf{RQ1:} How faster request processing time of multiple Sun PDPs with policies splitted by our approach compared
to an existing single Sun PDP? This question helps to show that our approach can improve performance in terms of request processing time. 
Moreover, we compare request processing time for different splitting criteria.
\item \textbf{RQ2:} This research question discusses \textit{Hypothesis 2} (mentioned in section \ref{sec:context}): 
With comparable PDP sizes, the overall system will be more performant when the architecture is synergic. 
Impact of synergy on system performance is discussed in this section.
\item \textbf{RQ3:} How faster request processing time of XEngine compared
to that of Sun PDP for both a single policy and policies split by our approach.
This question helps to show that our approach can improve performance in terms of request processing time for other policy evaluation 
engines such as XEngine.
\item \textbf{RQ4:} This research question is related to the effectiveness of the \textit{Hypothesis 1} 
(mentioned in section \ref{sec:context}): the bigger the PDP, the higher the slope of the execution 
time with an increasing workload. We aim to study the impact of the number of rules in a given PDP on system response.
\end{enumerate}

To address these research questions, we go through the following empirical protocols based on two different empirical studies:
\begin{itemize}
\item First, we evaluate the performance improvement regarding the decision making process by taking into consideration the whole system 
(PEPs and PDPs), We compared request processing time with a single global policy (handled by a single PDP) against request processing time with
split policies. All the splitting criteria have been considered in our evaluation.
\CodeIn{IA} denotes an ``Initial Architecture'', which uses the single global policy for request processing time.
This step allows studying the behavior of splitting criteria that preserve the synergy property in the access control architecture.

\item Second, we evaluate the performance of PDPs in isolation to compare the splitting criteria independently from the system.
Recall that in this step, we don't reason about the synergy property since we don't consider the application level.
We replace Sun PDPs with a novel request evaluation engine, XEngine \cite{Xengine}. 
The objective of this evaluation is to investigate how our approach impacts performance for subjects combined with XEngine.
\end{itemize}

\subsection{Subjects}
The subjects include three real-life Java programs each which interacts with access control policies \cite{testcase}. We next describe
our three subjects.
\begin{itemize}	
\item Library Management System (LMS) provides web services to manage books in a public library.
\item Virtual Meeting System (VMS) provides web conference services. VMS allows users to organize
online meetings in a distributed platform.
\item Auction Sale Management System (ASMS) allows users to buy or sell items on line. A seller 
initiates an auction by submitting a description of an item she wants to sell with its expected minimum 
price. Users then participate in bidding process by
bidding the item. For the bidding on the item, users have enough money in her account before bidding.
\end{itemize}
Our subjects are initially built upon Sun PDP \cite{sunxacml} as a decision engine, which is a popularly used PDP to evaluate
 requests. Policies in LMS, VMS, and ASMS contain a total of 720, 945, and 1760 rules, respectively.
Moreover, to compare performance improvement over existing PDPs, we adopt XEngine (instead of Sun PDP) in our subjects to evaluate requests.
XEngine is a novel policy evaluation engine, which transforms the hierarchical tree structure of the XACML policy to a flat structure
to improve request processing time. XEngine also handles various combining algorithms supported by XACML. 

\subsection{Performance Improvement: Sun PDP}\label{subsec:performanceimprovement}
In order to answer \textbf{RQ1}, we generated the resulting sub-policies for all the splitting criteria defined in 
section~\ref{subsec:SplittingCriteria}.s
For each splitting criteria, we have conducted systems tests to generate requests that trigger all the PEPs in the evaluation. 
The test generation step leads to the execution of all combination of possible requests described in our previous work \cite{testcase}.  
The process of test generation is repeated ten times to avoid the impact of randomness.
We applied this process to each splitting criterion and calculated evaluation time on average of a system under test.
% We only consider the execution time of the PDP and we do not include the executions of the system functions. 
Figure \ref{fig:processing time} presents evaluation time for policies split
based on each splitting criterion and the global policy of subjects. We can make two observations:
\begin{itemize}
\item Compared to evaluation time of \CodeIn{IA}, our approach improves performance for all of splitting criteria
in terms of evaluation time. This observation is consistent with our expected results; evaluation time against
policies with smaller number of rules (compared with the number of rules in \CodeIn{IA}) is faster than that against
policies in \CodeIn{IA}.
\item The splitting criteria \normalsize $SC=\langle Action, Resource\rangle$ enables to show the lowest evaluation time. 
Recall that the PEPs in the evaluation are scattered across different methods in a subject by a categorization 
that is based on $SC_{2}=\langle Resource,Action\rangle$. This observation pleads in favor of applying a splitting criteria 
that takes into account the PEP-PDP synergy.
\end{itemize}
 \begin{figure*}[h!]
  \centering
  \subfloat[LMS]{\label{fig:gull}\includegraphics[width=0.33\textwidth]{LMS.pdf}}                
  \subfloat[VMS]{\label{fig:VMS}\includegraphics[width=0.33\textwidth]{VMS.pdf}}
  \subfloat[ASMS]{\label{fig:ASMS}\includegraphics[width=0.33\textwidth]{ASMS.pdf}}
  \caption{Request Processing Time for our subjects LMS, VMS and ASMS}
  \label{fig:processing time}
\end{figure*}
 

To identify the splitting criterion that generates less number of PDPs, we have studied the number of policies generated by the
 splitting. Figure \ref{pdpnumber} shows the results.
We observed the number of policies based on our proposed three categories: (1) $SC_{1}$ category leads to the smallest number $N_1$ of PDPs, 
(2) $SC_{2}$ category produces a reasonable number
 $N_2$ ($N_1$<$N_2$<$N_3$) of PDPs, and (3) $SC_{3}$ leads to the largest number $N_3$ of PDPs.
While $SC_{1}$ category leads to the smallest number of PDPs, each PDP encapsulates a relatively high number of rules in a policy (compared
with that of $SC_{2}$ and $SC_{3}$, which leads to performance degradation. 

\begin{figure}[!h]
\centering
\includegraphics[width=8.5cm, height=7.2cm]{pdpnumber.pdf}
\begin{center}
\caption{PDP Number According to Splitting Criteria}
\label{pdpnumber}
\end{center}
\end{figure}

We have classified splitting criteria according to their preservation of the synergy property like shown in 
Table \ref{Classification}. $AR$, $A$, $R$ are synergic splitting criteria since all the PEPs in our three systems 
are organized like shown in Figure \ref{PEPdeploymentexample}. 

\begin{table}[t]
\centering
\begin{tabular}{|>{\tiny}c|>{\tiny}c|>{\tiny}c|>{\tiny}c|>{\tiny}c|>{\tiny}c|>{\tiny}c|>{\tiny}c|>{\tiny}c|}   
\hline  \rowcolor{black} \scriptsize \bf \textcolor {white}{}
& \scriptsize \bf \textcolor {white}{S}
& \scriptsize \bf \textcolor {white}{A}
& \scriptsize \bf \textcolor  {white}{R}
& \scriptsize \bf \textcolor  {white}{SA}
& \scriptsize \bf \textcolor  {white}{SR}
& \scriptsize \bf \textcolor  {white}{AR} 
& \scriptsize \bf \textcolor  {white}{SAR}
& \scriptsize \bf \textcolor {white}{IA}\\ \hline
\scriptsize  {Synergic}
&\scriptsize  {}
& \scriptsize {x}
& \scriptsize {x}
& \scriptsize {}
& \scriptsize {}
& \scriptsize {x}
& \scriptsize {}
& \scriptsize {x}
  \\ \hline
\scriptsize  {Not Synergic}
&\scriptsize  {x}
& \scriptsize {}
& \scriptsize {}
& \scriptsize {x}
& \scriptsize {x}
& \scriptsize {}
& \scriptsize {x}
& \scriptsize {}
  \\ \hline
\end{tabular}
\caption{Splitting Criteria Classification}
\label{Classification}
\end{table}

To answer \textbf{RQ2}, we have evaluated PDPs in the three systems and for the different splitting criteria. The results presented in Figure 
\ref{average} show the average number of rules in each PDP, for each splitting criterion in the three systems. We can observe that $AR$ criterion produces comparable size PDPs with $SR$ criterion, however, as shown in Figure 
\ref{fig:processing time}, $AR$ is the best splitting criterion, in term of evaluation time performance. 
Moreover, the number of PDPs produced with the splitting criterion $S$ and $R$ is comparable, the criterion 
$R$ which is synergic, has an evaluation time which is less than the one produced by the splitting criterion $S$ which is not synergic.
This strengthens our hypothesis \textit{Hypothesis 2} which states that with comparable PDP sizes, the overall system will 
be more performant when the architecture is synergic.

\begin{figure}[!h]
\centering
\includegraphics[width=8.5cm, height=7.2cm]{averagerules.pdf}
\begin{center}
\caption{Average of rules number/PDP in the 3 systems}
\label{average}
\end{center}
\end{figure}

%%%%%%%%%%%%%%%%%%%%%%%%%%%%%%%%%%%%%%%%%%%%%%%%%%%%%%%%%%%%%%%%%%%%%%%%%%%%%%%%%%%%%%%%%%%%%%%%%%%%%%%
\subsection{Performance Improvement:XEngine}
In order to answer \textbf{RQ3}, we measure request processing time of XEngine compared with that of Sun PDP
for policies split by our approach.
The goal of this empirical study is to show the impact of combining
XEngine with our splitting process. XEngine improves dramatically the performance of the PDP, mainly for
3 reasons:
\begin{itemize}
\item It uses a refactoring process that transforms the hierarchical structure of the XACML policy to a flat structure.
\item  It converts multiple combining algorithms to single one.
\item  It relies on a tree structure that minimizes the request
processing time.
\end{itemize}
We propose to use XEngine conjointly with the refactoring
process presented in this work: We have evaluated our
approach in two settings:
\begin{itemize}
\item  Considering evaluation with a decision engine, based
on Sun PDP, with split policies and with the initial policy.
\item Considering evaluation with a decision engine, based
on XEngine rather than Sun PDP, with split policies
and with the initial policy as well.
\end{itemize}

In this step, we do not reason about the synergy, since we do not consider the application level for the three systems.
We measure request processing time by evaluating randomly
generated set of 10,000 requests like proposed in our previous work \cite{request}.
The request time processing is evaluated for LMS, VMS, ASMS. The results
are presented in Table \ref{table:LMSeval}, \ref{table:VMSeval} and \ref{table:ASMSeval} and enable to answer \textbf{RQ3}.
We observe that, in case our subjects are equipped with XEngine, our proposed approach 
improves performance (compared to the results with Sun PDP). For the splitting criteria SC=$\langle Action \rangle$,
in the LMS system, the evaluation time is reduced about 9 times: from 2703 ms to 290 ms with XEngine. This empirical 
observation pleads in favour of applying our proposed refactoring process with XEngine 

\begin{table}[t]
\centering
\begin{tabular}{|>{\tiny}c|>{\tiny}c|>{\tiny}c|>{\tiny}c|>{\tiny}c|>{\tiny}c|>{\tiny}c|>{\tiny}c|>{\tiny}c|}   
\hline  \rowcolor{black} \scriptsize \bf \textcolor {white}{}
& \scriptsize \bf \textcolor {white}{SAR}
& \scriptsize \bf \textcolor {white}{AR}
& \scriptsize \bf \textcolor  {white}{SA}
& \scriptsize \bf \textcolor  {white}{SR}
& \scriptsize \bf \textcolor  {white}{R}
& \scriptsize \bf \textcolor  {white}{S} 
& \scriptsize \bf \textcolor  {white}{A}
& \scriptsize \bf \textcolor {white}{IA}\\ \hline
\scriptsize  {Sun PDP }
&\scriptsize  {485}
& \scriptsize {922}
& \scriptsize {1453}
& \scriptsize {1875}
& \scriptsize {2578}
& \scriptsize {2703}
& \scriptsize {2703}
& \scriptsize {2625}
  \\ \hline
\scriptsize  {XEngine}
&\scriptsize  {26}
& \scriptsize {47}
& \scriptsize {67}
& \scriptsize {95}
& \scriptsize {190}
& \scriptsize {164}
& \scriptsize {120}
& \scriptsize {613}
  \\ \hline
\end{tabular}

\caption{Evaluation time in LMS}
\label{table:LMSeval}
\vspace{5 mm}
%\end{table}
%\begin{table}[h!]
\centering
\begin{tabular}{|l|l|l|l|l|l|l|l|l|}   
\hline  \rowcolor{black} \scriptsize \bf \textcolor {white}{}
& \scriptsize \bf \textcolor {white}{SAR}
& \scriptsize \bf \textcolor {white}{AR}
& \scriptsize \bf \textcolor  {white}{SA}
& \scriptsize \bf \textcolor  {white}{SR}
& \scriptsize \bf \textcolor  {white}{R}
& \scriptsize \bf \textcolor  {white}{S} 
& \scriptsize \bf \textcolor  {white}{A}
& \scriptsize \bf \textcolor {white}{IA}\\ \hline
\scriptsize  {Sun PDP }
& \scriptsize  {1281}
& \scriptsize {2640}
& \scriptsize {3422}
& \scriptsize {3734}
& \scriptsize {6078}
& \scriptsize {5921}
& \scriptsize {6781}
& \scriptsize {5766}
  \\ \hline
\scriptsize  {XEngine}
& \scriptsize  {34}
& \scriptsize {67}
& \scriptsize {96}
& \scriptsize {145}
& \scriptsize {384}
& \scriptsize {274}
& \scriptsize {149}
& \scriptsize {265}
  \\ \hline
\end{tabular}

\caption{Evaluation time in VMS}
\label{table:VMSeval}
\vspace{5 mm}
\centering
\begin{tabular}{|l|l|l|l|l|l|l|l|l|}   
\hline  \rowcolor{black} \scriptsize \bf \textcolor {white}{}
& \scriptsize \bf \textcolor {white}{SAR}
& \scriptsize \bf \textcolor {white}{AR}
& \scriptsize \bf \textcolor  {white}{SA}
& \scriptsize \bf \textcolor  {white}{SR}
& \scriptsize \bf \textcolor  {white}{R}
& \scriptsize \bf \textcolor  {white}{S} 
& \scriptsize \bf \textcolor  {white}{A}
& \scriptsize \bf \textcolor {white}{IA}\\ \hline
\scriptsize  {Sun PDP  }
& \scriptsize  {2280}
& \scriptsize {2734}
& \scriptsize {3625}
& \scriptsize {8297}
& \scriptsize {7750}
& \scriptsize {8188}
& \scriptsize {6859}
& \scriptsize {7156}
  \\ \hline
\scriptsize  {XEngine}
& \scriptsize  {49}
& \scriptsize {60}
& \scriptsize {104}
& \scriptsize {196}
& \scriptsize {310}
& \scriptsize {566}
& \scriptsize {262}
& \scriptsize {1639}
  \\ \hline
\end{tabular}
\caption{Evaluation time in ASMS}
\label{table:ASMSeval}
\end{table}
%%%%%%%%%%%%%%%%%%%%%%%%%%%%%%%%%%%%%%%%%ùùùùù
\subsection{Impact of increasing workload}\label{subsec:Systemworkload}
To investigate \textbf{RQ4}, we evaluated request processing time according to the number of requests incoming to the system. 
For each policy in the three systems (ASMS, LMS, and VMS), we generated 5000, 10000, .., 50000 random requests to measure the evaluation time (ms).
The results are shown in Figure \ref{fig:processing time xengine}. For the three systems, we notice that the evaluation time increases when 
the number of requests increases in the system. With an increasing system load, the request evaluation time is considerably
 improved when using the splitting process compared to the initial architecture. 
The results shown in Figure \ref{fig:processing time xengine} are interpreted by the average of PDPs size presented in 
Figure \ref{average}, which is consistent with \textit{Hypothesis 1} (mentioned in section \ref{sec:context}) 
which states that the slope of execution time increases with PDPs size in a system with an increasing workload.
\begin{figure*}
  \centering
  \subfloat[LMS]{\label{fig:gull}\includegraphics[width=0.34\textwidth]{X-LMS.pdf}}                
  \subfloat[VMS]{\label{fig:VMS}\includegraphics[width=0.34\textwidth]{X-VMS.pdf}}
  \subfloat[ASMS]{\label{fig:ASMS}\includegraphics[width=0.34\textwidth]{X-ASMS.pdf}}
  \caption{Processing Time for our subjects, LMS, VMS and ASMS depending on the requests Number}
  \label{fig:processing time xengine}
\end{figure*}

It is worth to mention that, to be able to deploy our technique, we need to fetch the relevant PDP for a given request at runtime. 
Therefore, request processing time includes both fetching time and request evaluation time.
Figure \ref{Fetching Time} shows percentage of fetching time over the global evaluation time for request evaluation in LMS. 
The fetching time increases according to the PDPs size. The fetching time is relatively small in comparison with the total evaluation time and thus 
does not impact significantly the decision making time.
\begin{figure}[!h]
  \centering
\includegraphics[width=8.5cm, height=7.2cm]{fetching.pdf}
\begin{center}
\caption{Percentage of Fetching Time}
\label{Fetching Time}
\end{center}
\end{figure}
\subsection{Discussion}
We begin the discussion by summarizing the results of the evaluation section:
\begin{itemize}
 \item We have experimentally shown the effectiveness of the splitting in reducing the decision making time. Our refactoring process is more efficient when 
it uses XEngine rather than Sun PDP to evaluate requests.
 \item When the size of PDPs are comparable, the splitting criteria that are synergic enable to have the best results in terms of 
decision making time.
\end{itemize}
The effectiveness of the synergy property on performance has to be strengthened by conductinng others experiments 
on other evaluation studies and by considering different organizations of PEPs at the application level.